\begin{abstract}

Over the past one billion years, Earth's climate has fluctuated between three stable states on million year time-scales: a warm state in which the poles are ice-free, a cold state in which finite ice caps exist at the poles, and a ``snowball'' state in which Earth's entire surface is covered by ice. Changes in global weatherability could be responsible for driving transitions between these climate states by modulating the atmospheric \COtwo concentration (\pCOtwo) at which \COtwo input from volcanism into Earth's ocean/atmosphere system is removed via silicate weathering. Since the presence of mafic and ultramafic rocks in the warm and wet tropics increases global weatherability, it has both been hypothesized that island arc exhumation and large igneous province eruption at low latitudes have driven cooling on million year time-scales. In the chapters presented in this dissertation, we evaluate these two hypotheses.

In Chapter 1, we reconstruct the paleogeographic position of major arc-continent collisions and large igneous provinces to assess whether a first-order correlation between these two tectonic settings and changes in Earth's climate state can be established for the Phanerozoic. Arc-continent collisions are quantified as the length of sutures that are active at any given time, and large igneous provinces are quantified as the area of surface volcanics remaining following eruption after a parameterization of erosion has been applied. The latitudinal distribution of continental ice sheets is used as a proxy for Earth's climate state. Our analyses reveal a strong correlation between active suture length in the tropics and the extent of glaciation, and no significant correlation between large igneous province area in the tropics and the extent of glaciation. The key difference between large igneous provinces and active orogens involving island arcs is continuous exhumation and the creation of steep topography in the orogens. Therefore, our results suggest that changes in Earth's climate state are primarily driven by island arc exhumation in the tropics due to the combination of mafic and ultramafic lithologies, a warm and wet tropical environment, high erosion rates, and a lack of thick regoliths in this tectonic setting. In contrast, large igneous provinces have low erosion rates and develop thick regoliths, dampening their influence on global weatherability and Earth's climate state.

However, this correlation between arc-continent collisions in the tropics and Earth's climate state over the Phanerozoic does not necessitate causation. The magnitude of decrease in steady-state \pCOtwo associated with specific instances of arc-continent collision in the tropics needs to be quantified. Ongoing arc–continent collision in the tropical Southeast Asian islands has increased the area of subaerially exposed land in the region since the mid-Miocene. Concurrently, Earth’s climate has cooled since the Miocene Climatic Optimum, leading to growth of the Antarctic ice sheet and the onset of Northern Hemisphere glaciation. In Chapter 2, we compile paleoshoreline data and incorporate them into a numerical model that couples a global climate model to a silicate weathering model with spatially resolved lithology. We find that without the increase in area of the Southeast Asian islands over the Neogene, \pCOtwo would have been significantly higher than pre-industrial values, remaining above the levels necessary for initiating Northern Hemisphere ice sheets.

As such, there is accumulating evidence that supports the notion that transitions between ice-free and ice-cap climate states is primarily driven by island arc exhumation in the tropics. However, it remains unclear whether transitions into the snowball climate state are driven by the same mechanism. In Chapter 3, we investigate the Tonian-Cryogenian Tambien Group of northern Ethiopia -- a mixed carbonate-siliciclastic sequence that culminates in glacial deposits associated with the ca. 717--660~Ma Sturtian ``Snowball Earth.'' The presence of intercalated tuffs suitable for high-precision geochronology within the Tambien Group enable temporal constraints on stratigraphic data sets of the interval preceding, and leading into, the Sturtian glaciation. \dC and \SrSr data and U-Pb chemical abrasion isotope dilution thermal ionization mass spectrometry (CA-ID-TIMS) ages from the Tambien Group are used in conjunction with previously published isotopic and geochronologic data to construct newly time-calibrated composite Tonian carbon and strontium isotope curves. Tambien Group \dC data and U-Pb CA-ID-TIMS ages reveal that a pre-Sturtian sharp negative \dC excursion precedes the Sturtian glaciation by $\sim$18~Myr and is followed by a prolonged interval of positive \dC values, suggesting that perturbations to the carbon cycle that lead to sharp negative \dC excursions are unrelated to the initiation of the Sturtian glaciation. The composite Tonian \SrSr curve shows that, following an extended interval of low and relatively invariant values, inferred seawater \SrSr rose ca. 880--770~Ma, and then decreased to the ca. 717~Ma initiation of the Sturtian glaciation. These data, when combined with a simple global weathering model and analyses of the timing and paleolatitude of large igneous province eruptions and island arc exhumation events, suggest that the \SrSr increase was influenced by increased subaerial weathering of radiogenic lithologies as the (super)continent Rodinia rifted apart at low latitudes. The following \SrSr decrease is consistent with enhanced subaerial weathering of island arc lithologies accreting in the tropics over tens of millions of years, lowering \pCOtwo and contributing to the initiation of the Sturtian glaciation.

However, a lack of paleomagnetic data to constrain the paleolatitude and configuration of tectonic blocks during the Tonian hampers efforts to quantify changes in global weatherability during the lead up to the Sturtian glaciation. South China is associated with arc-continent collisions during the Tonian, and is at the center of debates regarding the configuration of Rodinia, with competing models variably placing the block at the core or periphery of Rodinia, or separated from it entirely. Tonian paleogeographic models also vary in whether they incorporate proposed large and rapid oscillatory true polar wander associated with the ca. 810--795~Ma Bitter Springs Stage. In Chapter 4, we develop new paleomagnetic data paired with U-Pb CA-ID-TIMS geochronology from the Tonian Xiajiang Group in South China to establish the block's position and test the Bitter Springs Stage true polar wander hypothesis. The data constrain South China to high latitudes ca. 813~Ma, and when considered in conjunction with other paleomagnetic poles from South China, indicate a relatively stable high-latitude position ca. 821--805~Ma. The difference in pole position between the pre-Bitter Springs Stage Xiajiang Group pole and the syn-Bitter Springs Stage Madiyi Formation pole is significantly less than that predicted for the Bitter Springs Stage true polar wander hypothesis. These constraints place the craton at higher latitudes connected to Rodinia along its periphery, or disconnected from Rodinia entirely. If this pole difference is interpreted as true polar wander superimposed upon differential plate motion, it requires South China to have been separate from Rodinia.

Put together, we find that the exhumation of island arcs and oceanic crust during arc-continent collision and arc-accretion events in the tropics are important for driving shifts in Earth's climate state over the past one billion years.

\end{abstract}
