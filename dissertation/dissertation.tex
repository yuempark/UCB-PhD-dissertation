% Official and up-to-date UC Berkeley guidelines on
% dissertation formatting can be found here:
% https://grad.berkeley.edu/academic-progress/dissertation/

%%%%%%%%%%%%%%%%%%%%
%%%%% PACKAGES %%%%%
%%%%%%%%%%%%%%%%%%%%

\documentclass{ucbthesis}
%\usepackage{biblatex}
\usepackage{natbib} % use natbib over biblatex to be able to use GSAB .bst
\usepackage{textcomp,marvosym}
\usepackage{amsmath,amssymb}
\usepackage{rotating}
\usepackage{graphicx}
\usepackage{xspace}
\usepackage[hidelinks]{hyperref}
\urlstyle{same}
\usepackage{threeparttable}
\usepackage[font=footnotesize,format=plain,labelfont=bf,up,textfont=up]{caption} % change figure caption style
\usepackage{color,colortbl}
\usepackage{bibentry}
\bibliographystyle{gsabull}

% To compile this file, run "latex thesis", then "biber thesis"
% (or "bibtex thesis", if the output from latex asks for that instead),
% and then "latex thesis" (without the quotes in each case).

% Double spacing, if you want it. Do not use for the final copy.
% \def\dsp{\def\baselinestretch{2.0}\large\normalsize}
% \dsp

% If the Grad. Division insists that the first paragraph of a section
% be indented (like the others), then include this line:
% \usepackage{indentfirst}

\addtolength{\abovecaptionskip}{\baselineskip}

% Not sure why, but this is required by the style class
\newtheorem{theorem}{Jibberish}

% set hyphenation rules if words are not hyphenated properly
%\hyphenation{mar-gin-al-ia}

%%%%%%%%%%%%%%%%%%%
%%%%% SYMBOLS %%%%%
%%%%%%%%%%%%%%%%%%%

\newcommand{\degreesC}{$^{\circ}$C\xspace}
\newcommand{\degrees}{$^{\circ}$\xspace}
\newcommand{\dC}{$\delta^{13}$C\xspace}
\newcommand{\dO}{$\delta^{18}$O\xspace}
\newcommand{\SrSr}{$^{87}$Sr/$^{86}$Sr\xspace}
\newcommand{\permil}{\textperthousand\xspace}
\newcommand{\dsil}{$d$\xspace}
\newcommand{\UPb}{$^{206}$Pb/$^{238}$U\xspace}
\newcommand{\pCOtwo}{$p$CO$_{2}$\xspace}
\newcommand{\COtwo}{CO$_{2}$\xspace}
\newcommand{\SI}{\textit{Supporting Information}\xspace}
\newcommand{\MM}{\textit{Materials and Methods}\xspace}

\definecolor{Yellow}{rgb}{1,1,0.35}

%%%%%%%%%%%%%%%%%%%%%%%%
%%%%% FRONT MATTER %%%%%
%%%%%%%%%%%%%%%%%%%%%%%%

\begin{document}
\nobibliography* % enigmatically required to use bibentry

\title{Planetary cooling, tectonics, and weathering from 1 billion years ago to the present}

\author{Yuem Park}
\degreesemester{Fall}
\degreeyear{2020}
\degree{Doctor of Philosophy}
\chair{Professor Nicholas L. Swanson-Hysell}
\othermembers{Professor Daniel A. Stolper \\
			  Professor Seth Finnegan}
% For a co-chair who is subordinate to the \chair listed above
% \cochair{Professor Benedict Francis Pope}
% For two co-chairs of equal standing (do not use \chair with this one)
% \cochairs{Professor Richard Francis Sony}{Professor Benedict Francis Pope}
\numberofmembers{3}
\field{Earth \& Planetary Science}
% This is optional (default is Berkeley)
% \campus{Berkeley}

\maketitle
% Delete (or comment out) the \approvalpage line for the final version.
\approvalpage
\copyrightpage

\begin{abstract}

Over the past one billion years, Earth's climate has fluctuated between three stable states on million year time-scales: a warm state in which the poles are ice-free, a cold state in which finite ice caps exist at the poles, and a ``snowball'' state in which Earth's entire surface is covered by ice. Changes in global weatherability could be responsible for driving transitions between these climate states by modulating the atmospheric \COtwo concentration (\pCOtwo) at which \COtwo input from volcanism into Earth's ocean/atmosphere system is removed via silicate weathering. Since the presence of mafic and ultramafic rocks in the warm and wet tropics increases global weatherability, it has both been hypothesized that island arc exhumation and large igneous province eruption at low latitudes have driven cooling on million year time-scales. In the chapters presented in this dissertation, we evaluate these two hypotheses.

In Chapter 1, we reconstruct the paleogeographic position of major arc-continent collisions and large igneous provinces to assess whether a first-order correlation between these two tectonic settings and changes in Earth's climate state can be established for the Phanerozoic. Arc-continent collisions are quantified as the length of sutures that are active at any given time, and large igneous provinces are quantified as the area of surface volcanics remaining following eruption after a parameterization of erosion has been applied. The latitudinal distribution of continental ice sheets is used as a proxy for Earth's climate state. Our analyses reveal a strong correlation between active suture length in the tropics and the extent of glaciation, and no significant correlation between large igneous province area in the tropics and the extent of glaciation. The key difference between large igneous provinces and active orogens involving island arcs is continuous exhumation and the creation of steep topography in the orogens. Therefore, our results suggest that changes in Earth's climate state are primarily driven by island arc exhumation in the tropics due to the combination of mafic and ultramafic lithologies, a warm and wet tropical environment, high erosion rates, and a lack of thick regoliths in this tectonic setting. In contrast, large igneous provinces have low erosion rates and develop thick regoliths, dampening their influence on global weatherability and Earth's climate state.

However, this correlation between arc-continent collisions in the tropics and Earth's climate state over the Phanerozoic does not necessitate causation. The magnitude of decrease in steady-state \pCOtwo associated with specific instances of arc-continent collision in the tropics needs to be quantified. Ongoing arc–continent collision in the tropical Southeast Asian islands has increased the area of subaerially exposed land in the region since the mid-Miocene. Concurrently, Earth’s climate has cooled since the Miocene Climatic Optimum, leading to growth of the Antarctic ice sheet and the onset of Northern Hemisphere glaciation. In Chapter 2, we compile paleoshoreline data and incorporate them into a numerical model that couples a global climate model to a silicate weathering model with spatially resolved lithology. We find that without the increase in area of the Southeast Asian islands over the Neogene, \pCOtwo would have been significantly higher than pre-industrial values, remaining above the levels necessary for initiating Northern Hemisphere ice sheets.

As such, there is accumulating evidence that supports the notion that transitions between ice-free and ice-cap climate states is primarily driven by island arc exhumation in the tropics. However, it remains unclear whether transitions into the snowball climate state are driven by the same mechanism. In Chapter 3, we investigate the Tonian-Cryogenian Tambien Group of northern Ethiopia -- a mixed carbonate-siliciclastic sequence that culminates in glacial deposits associated with the ca. 717--660~Ma Sturtian ``Snowball Earth.'' The presence of intercalated tuffs suitable for high-precision geochronology within the Tambien Group enable temporal constraints on stratigraphic data sets of the interval preceding, and leading into, the Sturtian glaciation. \dC and \SrSr data and U-Pb chemical abrasion isotope dilution thermal ionization mass spectrometry (CA-ID-TIMS) ages from the Tambien Group are used in conjunction with previously published isotopic and geochronologic data to construct newly time-calibrated composite Tonian carbon and strontium isotope curves. Tambien Group \dC data and U-Pb CA-ID-TIMS ages reveal that a pre-Sturtian sharp negative \dC excursion precedes the Sturtian glaciation by $\sim$18~Myr and is followed by a prolonged interval of positive \dC values, suggesting that perturbations to the carbon cycle that lead to sharp negative \dC excursions are unrelated to the initiation of the Sturtian glaciation. The composite Tonian \SrSr curve shows that, following an extended interval of low and relatively invariant values, inferred seawater \SrSr rose ca. 880--770~Ma, and then decreased to the ca. 717~Ma initiation of the Sturtian glaciation. These data, when combined with a simple global weathering model and analyses of the timing and paleolatitude of large igneous province eruptions and island arc exhumation events, suggest that the \SrSr increase was influenced by increased subaerial weathering of radiogenic lithologies as the (super)continent Rodinia rifted apart at low latitudes. The following \SrSr decrease is consistent with enhanced subaerial weathering of island arc lithologies accreting in the tropics over tens of millions of years, lowering \pCOtwo and contributing to the initiation of the Sturtian glaciation.

However, a lack of paleomagnetic data to constrain the paleolatitude and configuration of tectonic blocks during the Tonian hampers efforts to quantify changes in global weatherability during the lead up to the Sturtian glaciation. South China is associated with arc-continent collisions during the Tonian, and is at the center of debates regarding the configuration of Rodinia, with competing models variably placing the block at the core or periphery of Rodinia, or separated from it entirely. Tonian paleogeographic models also vary in whether they incorporate proposed large and rapid oscillatory true polar wander associated with the ca. 810--795~Ma Bitter Springs Stage. In Chapter 4, we develop new paleomagnetic data paired with U-Pb CA-ID-TIMS geochronology from the Tonian Xiajiang Group in South China to establish the block's position and test the Bitter Springs Stage true polar wander hypothesis. The data constrain South China to high latitudes ca. 813~Ma, and when considered in conjunction with other paleomagnetic poles from South China, indicate a relatively stable high-latitude position ca. 821--805~Ma. The difference in pole position between the pre-Bitter Springs Stage Xiajiang Group pole and the syn-Bitter Springs Stage Madiyi Formation pole is significantly less than that predicted for the Bitter Springs Stage true polar wander hypothesis. These constraints place the craton at higher latitudes connected to Rodinia along its periphery, or disconnected from Rodinia entirely. If this pole difference is interpreted as true polar wander superimposed upon differential plate motion, it requires South China to have been separate from Rodinia.

Put together, we find that the exhumation of island arcs and oceanic crust during arc-continent collision and arc-accretion events in the tropics are important for driving shifts in Earth's climate state over the past one billion years.

\end{abstract}


\begin{frontmatter}

%%%%%%%%%%%%%%%%%%%%%%%%%
%%%%% TOC AND LISTS %%%%%
%%%%%%%%%%%%%%%%%%%%%%%%%

\tableofcontents
\clearpage
\listoffigures
\clearpage
\listoftables

%%%%%%%%%%%%%%%%%%%
%%%%% PREFACE %%%%%
%%%%%%%%%%%%%%%%%%%

\begin{preface}
\phantomsection % make the ToC links work
\addcontentsline{toc}{chapter}{Preface} % add to ToC

The material within the chapters of this dissertation are largely taken from previously published (or soon to be published) articles. The following text identifies these articles, and provides some additional context about how they relate to one another.

\subsubsection*{Chapter 1 - Evaluating the relationship between the area and latitude of large igneous provinces and Earth's long-term climate state}

\noindent
\bibentry{Macdonald2019a}

\medskip

\noindent
\bibentry{Park2019a}

\bigskip

In \citet{Macdonald2019a}, we developed a workflow built on pyGPlates (software for the analysis of paleogeographic models) that allowed us to calculate the area/length of paleogeographically reconstructed polygons/lines within different latitude bands through time. By pairing this workflow with a database of ophiolite-bearing sutures and a paleogeographic model, we were able to identify a strong correlation between arc-continent collisions in the tropics (measured as active suture length in the tropics) and times of glacial climate over the past $\sim$520~m.y. This correlation led to our proposal that Earth's climate state is set primarily by global weatherability, which changes with the latitudinal distribution of arc-continent collisions. Similarly, it has been proposed elsewhere that the eruption/drift of large igneous provinces into the tropics has driven global cooling. The pyGPlates workflow that we developed provided a means through which we could evaluate this hypothesis. \citet{Park2019a} builds upon the zonal large igneous province area analysis presented in the supplementary materials for \citet{Macdonald2019a} by more rigorously developing parameterizations of large igneous province erosion and exploring several geologically reasonable large igneous province post-emplacement scenarios.

\citet{Park2019a} is currently available as a preprint on EarthArXiv, but a slightly modified version of the article is scheduled to be formally published within 2020 as Chapter 7 of AGU Geophysical Monograph Series 255, titled \textit{Large Igneous Provinces: A Driver of Global Environmental and Biotic Changes}, edited by Richard E. Ernst, Alexander J. Dickson, and Andrey Bekker.

Code used to perform the analyses presented in these articles are available on GitHub at \url{https://github.com/Swanson-Hysell-Group/Arc_Continent_Analysis} and \url{https://github.com/Swanson-Hysell-Group/2020_large_igneous_provinces}, or on Zenodo at \url{https://doi.org/10.5281/zenodo.2636731} and \url{https://doi.org/10.5281/zenodo.3981262}.

\subsubsection*{Chapter 2 - Emergence of the Southeast Asian islands as a driver for Neogene cooling}

\noindent
\bibentry{Park2020b}

\bigskip

This article was the result of a collaboration between Godd\'eris' and Swanson-Hysell's research groups supported by the France-Berkeley Fund. Broadly speaking, the collaboration was intended to investigate whether numerical Earth system models quantitatively support the hypothesis put forward in \citet{Macdonald2019a} and \citet{Swanson-Hysell2017a} that tropical arc-continent collisions set Earth's climate state. While the project that was initially proposed sought to investigate whether arc-continent collisions during the Taconic Orogeny led to glaciation in the Ordovician \citep{Swanson-Hysell2017a}, the project changed trajectory to instead investigate more recent changes in the Southeast Asian islands, when constraints on paleogeography are more robust. The collaboration between the two research groups is ongoing, with Maffre, a former Ph.D. candidate in Godd\'eris' group, currently working as a post-doctoral scholar in Swanson-Hysell's group at the time of writing.

The code for the GEOCLIM model used in this study can be found at \url{https://github.com/piermafrost/GEOCLIM-dynsoil-steady-state/releases/tag/v1.0}. The code that generated the inputs and analyzed the output of the GEOCLIM model can be found at \url{https://github.com/Swanson-Hysell-Group/2020_Southeast_Asian_Islands} or \url{https://doi.org/10.5281/zenodo.4021653}.

\subsubsection*{Chapter 3 - The lead-up to the Sturtian Snowball Earth: Neoproterozoic chemostratigraphy time-calibrated by the Tambien Group of Ethiopia}

\noindent
\bibentry{MacLennan2018a}

\medskip

\noindent
\bibentry{Park2020a}

\bigskip

Chronologically, the work that lead to \citet{MacLennan2018a} and \citet{Park2020a} largely preceded the projects presented in Chapters 1 and 2. In \citet{MacLennan2018a}, we presented the initial geochronologic results that placed temporal constraints on large excursions in the marine \dC record and the initiation of the Sturtian glaciation in a short-format journal. \citet{Park2020a} built upon this work by: thoroughly documenting the geology of the Tambien Group in northern Ethiopia, presenting additional geochronologic results, assessing diagenesis in carbonate samples, presenting the complete \dC and \SrSr data from the Tambien Group, developing a composite Tonian-Cryogenian \dC and \SrSr chemostratigraphic record, and proposing a model for the initiation of the Sturtian glaciation based on the composite \SrSr chemostratigraphic record, a simple global weathering model, and analyses of the timing and paleolatitude of large igneous province eruptions and arc accretion events. As in \citet{Swanson-Hysell2017a} and \citet{Macdonald2019a}, the model for the initiation of the Sturtian glaciation proposed in \citet{Park2020a} called upon arc-continent collisions in the tropics, and was part of the motivation to initiate a collaboration with Godd\'eris' research group.

Code used to perform the analyses presented in these articles is available on GitHub at \url{https://github.com/Swanson-Hysell-Group/2019_Tambien_Group}, or on Zenodo at \url{https://doi.org/10.5281/zenodo.3403180}.

\subsubsection*{Chapter 4 - Tonian paleomagnetism from South China permits either an inclusive Rodinia or Bitter Springs true polar wander, but not both}

\noindent
\textit{PRE-PRINT CITATION WILL GO HERE}
%\bibentry{Park2020c}

\bigskip

Although \citet{Park2020a} called upon arc-continent collisions in the tropics as a critical tectonic boundary condition that allowed for the Sturtian glaciation to take place, a lack of paleomagnetic data to constrain the paleolatitude and configuration of tectonic blocks during the lead up to the glaciation hampered the robust defense of this hypothesis. \citet{Park2020c} develops paleomagnetic data to better constrain the location of one of these tectonic blocks, South China, in the Tonian. South China is a particularly important block to investigate with paleomagnetic data in the context of the Sturtian glaciation because arc-continent collisions are hypothesized to have occurred both within and along the margin of the block during the Tonian. Furthermore, sediments from which paleomagnetic data can be developed exist in South China that span a hypothesized major global tectonic event (Bitter Springs Stage true polar wander) that, if real, has far-reaching implications for the paleogeography of the time.

\citet{Park2020c} is currently available as a preprint on EarthArXiv, and has been submitted to the \textit{Journal of Geophysical Research: Solid Earth} for review.

Code used to perform the analyses presented in these articles is available on GitHub at \url{xxx}, or on Zenodo at \url{xxx}.

\end{preface}

%%%%%%%%%%%%%%%%%%%%%%%%%%%%
%%%%% ACKNOWLEDGEMENTS %%%%%
%%%%%%%%%%%%%%%%%%%%%%%%%%%%

\begin{acknowledgements}
\phantomsection % make the ToC links work
\addcontentsline{toc}{chapter}{Acknowledgements} % add to ToC
Bovinely invasive brag; cerulean forebearance.
Washable an acre. To canned, silence in foreign.
Be a popularly. A as midnight transcript alike.
To by recollection bleeding. That calf are infant. In clause.
Buckaroo loquaciousness?  Aristotelian!
Masterpiece as devoted. My primal the narcotic. For cine?
In the glitter. For so talented. Which is confines cocoa accomplished.
Or obstructive, or purposeful.
And exposition? Of go. No upstairs do fingering.
\end{acknowledgements}

\end{frontmatter}

%%%%%%%%%%%%%%%%%%%%
%%%%% CHAPTERS %%%%%
%%%%%%%%%%%%%%%%%%%%

\pagestyle{headings}

% (Optional) \part{First Part}

\include{chap-LIPs}
\chapter[Emergence of the Southeast Asian islands as a driver for Neogene cooling][Southeast Asian islands]{Emergence of the Southeast Asian islands as a driver for Neogene cooling}

\section{Abstract}

Steep topography, a tropical climate, and mafic lithologies contribute to efficient chemical weathering and carbon sequestration in the Southeast Asian islands. Ongoing arc-continent collision between the Sunda-Banda arc system and Australia has increased the area of subaerially exposed land in the region since the mid-Miocene. Concurrently, Earth's climate has cooled since the Miocene Climatic Optimum, leading to growth of the Antarctic ice sheet and onset of Northern Hemisphere glaciation. We seek to evaluate the hypothesis that the emergence of the Southeast Asian islands played a significant role in driving this cooling trend through increasing global weatherability. To do so, we have compiled paleoshoreline data and incorporated them into GEOCLIM, which couples a global climate model to a silicate weathering model with spatially resolved lithology. We find that without the increase in area of the Southeast Asian islands over the Neogene, atmospheric \pCOtwo would have been significantly higher than pre-industrial values, remaining above the levels necessary for initiating Northern Hemisphere ice sheets.

\section{Introduction}

The Southeast Asian islands (SEAIs) have an out-sized contribution to modern chemical weathering fluxes relative to its area. The confluence of steep topography, a warm and wet tropical climate, and the presence of mafic lithologies results in high fluxes of Ca and Mg cations in the dissolved load and associated \COtwo consumption \citep{Gaillardet1999a, Hartmann2009a, Milliman2013a, Hartmann2014a}. There has been a significant increase in the area of subaerially exposed land within the region since the mid-Miocene associated with ongoing arc-continent collision between Australia and the Sunda-Banda arc system \citep{Molnar2015a, Hall2017a, Macdonald2019a}. Concurrently, after the Miocene Climatic Optimum, a cooling trend began ca. 15~Ma and accelerated over the past 4 million years (m.y.) leading to the development of Northern Hemisphere ice sheets \citep{Shackleton1984a, Zachos2001a}. Many hypotheses have been proposed to explain this cooling trend including changes in ocean/atmosphere circulation \citep{Haug1998a, Shevenell2004a, Molnar2015a}, a decrease in volcanic degassing \citep{Berner1983a}, or uplift in the Himalaya \citep{Raymo1988a, Galy2007a}. Here we seek to evaluate the hypothesis that emergence of the SEAIs was a significant factor in driving long-term climatic cooling over the Neogene.

Over geologic time-scales, \COtwo enters Earth's ocean--atmosphere system primarily via volcanism and metamorphic degassing, and leaves primarily through the chemical weathering of silicate rocks and through organic carbon burial \citep{Kump2000a}. Chemical weathering delivers alkalinity and cations to the ocean which drives carbon sequestration through carbonate precipitation. Steady-state \pCOtwo is set at the \pCOtwo level at which \COtwo sinks are equal to sources. As \COtwo sinks increase and \pCOtwo falls, temperature decreases and the hydrological cycle is weakened, causing the efficiency of the silicate weathering sink to decrease until a new steady-state is achieved at lower \pCOtwo \citep{Kump1997a}.

Topography, climate, and lithology all effect chemical weathering. High-relief regions generally lack extensive regolith development, and thus tend to have reaction-limited weathering regimes that are more prone to adjust when climate changes \citep{Gabet2009a, West2012a, Maher2014a}. High physical erosion rates contribute to high chemical weathering fluxes in these high-relief regions \citep{Godderis2017b}. In warm and wet regions, mineral dissolution kinetics are faster leading to enhanced chemical weathering \citep{Lasaga1994a, West2012a}. Mafic rocks have higher Ca and Mg concentrations and dissolution rates than felsic rocks, and thus have the potential to more efficiently sequester carbon through silicate weathering \citep{Dessert2003a}. These factors have led to the proposal that arc-continent collisions, which create steep landscapes that include mafic lithologies, within the tropical rain belt have been important in enhancing global weatherability, lowering atmospheric \pCOtwo, and initiating glacial climate over the past 520~m.y. \citep{Jagoutz2016a, Swanson-Hysell2017a, Macdonald2019a} and perhaps in the Neoproterozoic as well \citep{Park2020a}.

Quantitatively estimating the magnitude of decrease in steady-state \pCOtwo associated with the emergence of a region with a high carbon sequestration potential, such as the SEAIs, requires constraints on changing tectonic context and accounting of associated feedbacks. As this region emerges, the total global silicate weathering flux will transiently exceed the volcanic degassing flux, causing \pCOtwo to initially decline until a new steady-state is established where the total magnitude of the \COtwo sinks is the same as before the change. However, the sensitivity of the silicate weathering flux in any particular location to this change in \pCOtwo is variable and dependent on the specific topography, climate, and lithology at that location. Furthermore, how regional climate responds to this change in \pCOtwo is itself spatially variable. Therefore, the magnitude of \pCOtwo change that is required to balance the total global silicate weathering flux with the volcanic degassing flux will depend on the specific spatial distribution of topography, climate, and lithology at the time of emergence. As a result, any attempt to meaningfully estimate the decrease in steady-state \pCOtwo associated with emergence of the SEAIs must model spatially resolved climatology and silicate weathering fluxes in tandem and account for the spatial distribution of the factors that affect these inter-connected systems.

\section{GEOCLIM Model}

To estimate the decrease in steady-state \pCOtwo associated with the increase of subaerially exposed land area in the SEAIs, we use the global spatially resolved GEOCLIM model \citep{Godderis2017c}. GEOCLIM estimates changes in steady-state \pCOtwo associated with coupled changes in erosion, chemical weathering, and climatology by linking a silicate weathering model to climate model runs at multiple \pCOtwo levels.

\begin{figure}[h!]
\begin{center}
	\includegraphics[width=0.5\textwidth]{figures/SEAIs/regolith-schematic.pdf}
	\caption[Schematic representation of the silicate weathering component of GEOCLIM.]{A schematic representation of the silicate weathering component of GEOCLIM in a single profile at steady-state. A rock particle leaves the unweathered bedrock with production rate $P_{r}$, and transits vertically through a regolith of height $h$. Regolith production and physical erosion ($E_{p}$) are equal at steady-state. As a particle transits upwards, some fraction of the primary phases ($x$) are chemically weathered ($W$), with the flux of dissolved Ca+Mg being $W$ multiplied by the concentration of Ca+Mg in unweathered bedrock ($\chi_{CaMg}$). Details of the formulation for the silicate weathering component of GEOCLIM can be found in \MM.}
	\label{fig:regolith-schematic}
\end{center}
\end{figure}

The silicate weathering component of GEOCLIM calculates \COtwo consumption resulting from silicate weathering for subaerially exposed land. We assume that Ca and Mg are the only cations that consume \COtwo over geologic time-scales, such that each mole of Ca or Mg that is dissolved by silicate weathering consumes one mole of \COtwo. While reverse weathering is another potential sink for Ca or Mg \citep{Michalopoulos1995a}, its parameterization is unclear and it has been interpreted to be a relatively minor flux in the Cenozoic \citep{Isson2018a}, and we do not include it in our model. In previous versions of the model, silicate weathering was a function of temperature and runoff only, and all bedrock was assigned identical chemical compositions \citep{Godderis2017c}. More recent versions of GEOCLIM implement regolith development and soil shielding (Fig. \ref{fig:regolith-schematic}), which introduces a dependence on erosion rate (and therefore topographic slope; \citealp{Maffre2018a}). While this introduction of regolith development into GEOCLIM is important for assessing the impact of tropical arc-continent collisions on \pCOtwo, the relatively high Ca+Mg concentration in arc rocks relative to other lithologies must also be considered.

We therefore implement variable bedrock Ca+Mg concentration into GEOCLIM (\SI). The spatial distribution of lithologies is sourced from the Global Lithologic Map (GLiM; \citealp{Hartmann2012a}) and is represented by 6 categories: metamorphic, felsic, intermediate, mafic, carbonate, and siliciclastic sediment. Each land pixel is assigned these lithologic categories at a resolution of 0.1\degrees $\times$ 0.1\degrees. The Ca+Mg concentrations of felsic, intermediate, and mafic lithologies are assigned based on the mean of data of these lithologic categories compiled in EarthChem (\url{www.earthchem.org/portal}). Given that GLiM does not distinguish ultramafic lithologies, such rocks are grouped with mafic rocks. As a result, the Ca+Mg concentration is likely an underestimate in regions of obducted ophiolites, such that the estimated effect of these regions on changing steady-state \pCOtwo could be conservative \citep{Schopka2011a}. The weathering of carbonate does not contribute to long-term \COtwo consumption and its Ca+Mg concentration is ignored. The Ca+Mg concentrations of metamorphic and siliciclastic sediment lithologies are more difficult to define, since their chemical composition is strongly dependent on protolith composition and, in the case of siliciclastic sediment, the degree of previous chemical depletion. We explore a range of feasible Ca+Mg concentrations for metamorphic rocks and siliciclastic sediment during calibration of the silicate weathering component of GEOCLIM.

\subsection{Calibration}

The values of four parameters within the silicate weathering component that modify the dependence of silicate weathering on temperature, runoff, erosion, and regolith thickness are poorly constrained. Rather than prescribing single values, we select multiple values for each of these four parameters along with the Ca+Mg concentration of metamorphic and siliciclastic lithologies from within reasonable ranges (\SI; Table \ref{tab:parameter-combinations}). We then permute all possible combinations of these values for the six parameters, leading to 93,600 unique parameter combinations (i.e. permutations). For each combination, we compute spatially resolved long-term \COtwo consumption associated with Ca+Mg fluxes using present-day runoff, temperature, and slope. We sum computed \COtwo consumption over watersheds for which data-constrained estimates are available \citep{Gaillardet1999a, Moquet2018a}, then calculate the coefficient of determination ($r^{2}$) between computed and measured \COtwo consumption in each of these watersheds. After eliminating parameter combinations that result in low $r^{2}$, 573 parameter combinations remain (\SI; Fig. \ref{fig:W-vs-r2}). The resulting global \COtwo consumption of these filtered model runs all overlap with independently derived estimates of the global \COtwo degassing flux \citep{Gerlach2011a}, as they should for a steady-state long-term carbon cycle (\SI; Fig. \ref{fig:W-vs-r2}).

\subsection{Climate Model Component}

Having calibrated the silicate weathering component of GEOCLIM, we use it to estimate the decrease in steady-state \pCOtwo associated with emergence of the SEAIs. For the climate model component, we use temperature and runoff from a subset of the GFDL CM2.0 experiments (\SI; \citealp{Delworth2006b}). These experiments are well-suited for this analysis because all non-\COtwo forcings are held constant at values representative of pre-industrial conditions, allowing the effect of changing \pCOtwo on climatology to be isolated. Furthermore, the experiments were run long enough for the final system to approximate steady-state.

\section{Paleoshorelines}

\begin{figure}[h!]
\begin{center}
	\includegraphics[width=0.9\textwidth]{figures/SEAIs/shoreline-growth.pdf}
	\caption[Emergence of the Southeast Asian islands from the mid-Miocene to present.]{The emergence of the Southeast Asian islands (also referred to as the Maritime Continent in the climate science literature) from the mid-Miocene to present. Past shorelines at 5, 10, and 15 Ma are shown in \textbf{A} with associated land area summarized in \textbf{B}. A significant increase in area over the past 5 million years is coincident with cooling and the onset of Northern Hemisphere glaciation as reflected in the benthic oxygen isotope record \citep{Zachos2008a} shown in \textbf{C}.}
	\label{fig:shoreline-growth}
\end{center}
\end{figure}

\begin{figure}[h!]
\begin{center}
	\includegraphics[width=\textwidth]{figures/SEAIs/scenario-pCO2.pdf}
	\caption[Steady-state \pCOtwo estimates from GEOCLIM.]{Steady-state \pCOtwo estimates from GEOCLIM for the various scenarios discussed in the text. For each of the seven scenarios, each point represents an estimate from one of the 573 unique parameter combinations that most closely matched estimates of present-day \COtwo consumption in 80 watersheds around the world (\SI). The box encloses the middle 50\% of the \pCOtwo estimates (i.e. the interquartile range), and the notch represents the median with its 95\% confidence interval. The whiskers extend to the 2.5 and 97.5 percentile values. Glaciation thresholds \citep{DeConto2008a} are shown on the x-axis.}
	\label{fig:scenario-pCO2}
\end{center}
\end{figure}

To determine the position of paleoshorelines in the SEAIs over the past 15~m.y., we use terrestrial and marine sedimentary deposits (Fig. \ref{fig:shoreline-growth}; \SI). The paleoshoreline data indicate that the Sunda-Banda Arc and New Guinea are primarily responsible for the increase in area since 15~Ma. Exhumation of the modern Sunda-Banda Arc is the result of ongoing arc-continent collision with the Australian Plate \citep{Harris2006a}. Most of Sumatra and Java along with the non-volcanic islands of the Outer Banda Arc were elevated above sea level after 5~Ma \citep{Hall2013b}. In New Guinea, emergence in the mid-Miocene is associated with collision between the Melanesian Arc and Australia's distal margin \citep{Cloos2005a}, which drove exhumation of the Irian-Marum-April Ophiolite Belt. Exhumation accelerated over the past 4~m.y. in the New Guinea Central Range due to slab-breakoff and buoyant uplift, and in eastern New Guinea due to jamming of the north-dipping subduction zone \citep{Cloos2005a}. We also include changes in areas of presently submerged continental shelves such as the Sunda Shelf that were previously exposed (\SI; Fig. \ref{fig:SEAI-fracs}). These tectonic drivers and others throughout the region led to progressive emergence over the past 15~m.y. that accelerated following 5~Ma (Fig. \ref{fig:shoreline-growth}B). This trend mirrors broad cooling over the Neogene that resulted in the initiation of Northern Hemisphere ice sheets (Fig. \ref{fig:shoreline-growth}C).

We use GEOCLIM to estimate \pCOtwo associated with the reconstructed subaerial extent of the SEAIs at ca. 15, 10, and 5~Ma (``paleo-SEAIs'' scenarios; Fig. \ref{fig:scenario-pCO2}). Because we use a climate model forced with modern geography, the position of the tectonic blocks remain fixed. Although there has been motion of these tectonic blocks since 15~Ma, they have remained within tropical latitudes such that this fixed scenario is a good approximation of the paleogeography (\SI; Fig. \ref{fig:paleogeographic-reconstructions}). We also test an end-member scenario, in which all islands associated with arc-continent collision in the region are removed (``removed SEAIs'' scenario; Fig. \ref{fig:scenario-pCO2}).

\section{\pCOtwo Estimates}

Using the 573 unique parameter combinations, the paleo-SEAIs scenarios resulted in 526--678~ppm for 15~Ma, 457--516~ppm for 10~Ma, and 391--434~ppm \pCOtwo for 5~Ma (Fig. \ref{fig:scenario-pCO2}). These results indicate a progressive decrease in \pCOtwo over the Neogene associated with the emergence of the SEAIs, and suggest that without this emergence, pre-industrial \pCOtwo would have been $\sim$526--678~ppm. These paleo-SEAIs scenarios do not account for Neogene changes outside of the SEAIs (e.g. changes in ocean/atmosphere circulation, volcanic degassing, and weathering fluxes elsewhere on Earth, discussed in \textit{Alternative Mechanisms for Neogene Cooling}). Therefore, these results are not estimating \pCOtwo at 15~Ma, but rather are quantifying \pCOtwo change associated with emergence of the SEAIs.

Proxy-based estimates of the magnitude and trajectory of \pCOtwo change from the Miocene to the Pliocene are variable between techniques and associated assumptions underlying their interpretation (\SI; Fig. \ref{fig:pCO2-proxies}). The \pCOtwo values from the 5~Ma paleo-SEAIs scenario overlap with many proxy-based estimates \citep{Foster2017a} as well as values that emerge from approaches that assimilate climate and ice sheet model output with benthic \dO data \citep{van-de-Wal2011a, Berends2020a}. The modeled \pCOtwo values for 15~Ma resemble the higher end of proxy-based \pCOtwo estimates for the early to mid-Miocene, indicating that the increase in subaerially exposed land area and tectonic topography of the SEAIs is sufficient to explain long-term cooling of Earth's climate over the Neogene. The \pCOtwo threshold for Antarctic glaciation is estimated to be $\sim$750~ppm with that for Northern Hemisphere glaciation being significantly lower at $\sim$280~ppm \citep{DeConto2008a}. These modeled values of decreasing \pCOtwo associated with emergence of the SEAIs are therefore consistent with the record of Neogene climate with Miocene ice sheets on Antarctica \citep{Sugden1995a} followed by Northern Hemisphere ice sheets developing in the Pliocene \citep{Haug2005a} as \pCOtwo subsequently decreased.

The results of our paleo-SEAIs scenarios highlight the importance of the combination of topography, runoff, and lithology in setting Earth's climate state. To independently explore the effect of the modern-day surface exposure of lower-relief basaltic lavas on steady-state \pCOtwo \citep{Kent2013a}, we replace mafic volcanics associated with the Deccan Traps, Ethiopian Traps, and Columbia River Basalts with the Ca+Mg concentration of bulk continental crust in GEOCLIM (Fig. \ref{fig:scenario-pCO2}). The resulting \pCOtwo is $\sim$300--500~ppm, indicating that the presence of mafic rocks in these igneous provinces affects steady-state \pCOtwo as has been suggested to be important for Paleogene cooling \citep{Kent2013a}. However, the higher 526--678~ppm values for the 15~Ma paleo-SEAIs scenario illustrate that higher relief and a wet tropical climate significantly increase the efficiency of \COtwo consumption, especially when paired with high Ca+Mg lithologies. As such, arc-continent collisions in the tropics are likely more important for driving long-term changes in \pCOtwo than the eruption of flood basalts \citep{Macdonald2019a, Park2019a}.

Previous work has estimated that the decrease in \pCOtwo since 5~Ma associated with the emergence of the SEAIs and enhanced silicate weathering is $\sim$19~ppm \citep{Molnar2015a}, in which case their emergence would be a relatively minor contributor to Neogene cooling. This 19~ppm estimate was obtained using an equation that assumes a direct linear relationship between mean global temperature and changes in weathering-rate-weighted land area, scaled by a factor that is intended to account for the influence of both runoff and temperature. \pCOtwo was then estimated from the calculated temperature using a simple energy balance equation. However, the relationship between mean global temperature (or \pCOtwo) and weathering-rate-weighted land area is not linear. Furthermore, this simple linear relationship ignores spatial variability in topography and climatology, and only crudely accounts for spatial variability in lithology. In fact, the 19~ppm estimate is closer in magnitude to the decrease in \pCOtwo that we estimate if mafic volcanics associated with the Deccan Traps (a relatively flat area outside of the warm and wet tropics) are replaced with the Ca+Mg concentration of bulk continental crust (22--70 ppm; Fig. \ref{fig:scenario-pCO2}). The significant difference in steady-state \pCOtwo estimated between the ``removed Deccan Traps'' scenario and the paleo-SEAIs scenarios (Fig. \ref{fig:scenario-pCO2}) demonstrates that considering changes in the spatial distribution of lithologies alone is not adequate for estimating changes in steady-state \pCOtwo. Instead, spatially varying topography and climatology significantly modulates silicate weathering rates, and must be accounted for when estimating \pCOtwo change associated with paleogeographic change.

An important caveat for these estimates of \pCOtwo is that our modeling is determining the climatology in the GFDL CM2.0 model at which steady-state is achieved -- a climatology that has an associated \pCOtwo value in the model. However, climate models are variable in their response to changes in \pCOtwo. One way to summarize this variability is through the equilibrium climate sensitivity value -- the steady-state change in global mean surface air temperature associated with a doubling of \pCOtwo. A range of 1.5 to 4.5\degreesC per \pCOtwo doubling was proposed in the landmark Charney report \citep{Charney1979a} and this range was considered to be the credible interval (\textgreater66\% likelihood) in the last IPCC report \citep{Stocker2013a}. Integrating constraints both from understanding of climate feedback processes and the climate record, a recent comprehensive review estimates the 66\% probability range of climate sensitivity to be 2.6 to 3.9\degreesC per \pCOtwo doubling with a 5 to 95\% range of 2.3 to 4.7\degreesC per \pCOtwo doubling \citep{Sherwood2020a}. The equilibrium climate sensitivity associated with the CM2.0 climate models is 2.9\degreesC per \pCOtwo doubling, which falls within these ranges although these ranges remain broad. An alternative way to consider the results from our analysis would be that an estimate of 572~ppm (2$\times$ pre-industrial \pCOtwo) for the 15~Ma paleo-SEAIs scenario is implying that Earth would be $\sim$2.9\degreesC warmer. If Earth's climate sensitivity is at the higher end of the probable range and higher than in the CM2.0 model, as it is in some climate models, this same amount of Neogene cooling resulting from the emergence of the SEAIs could have been driven by a smaller change in \pCOtwo.

\section{Alternative Mechanisms for Neogene Cooling}

\subsection{Ocean/Atmosphere Circulation}

Some hypotheses to explain ice sheet growth over the Neogene invoke changes in ocean/atmosphere circulation including: further climatic isolation of Antarctica due to strengthening of the circumpolar current \citep{Shevenell2004a}; increased atmospheric moisture in the Northern Hemisphere due to intensified thermohaline circulation following Panama Isthmus emergence \citep{Haug1998a}; and cooling of North America resulting from a strengthened Walker Circulation associated with emergence of the SEAIs \citep{Molnar2015a}. Such changes in ocean/atmosphere circulation are likely to modulate \pCOtwo thresholds for glacial initiation and ice sheet growth \citep{DeConto2008a}. However, the prolonged time-scale of the cooling trend since 15~Ma (Fig. \ref{fig:shoreline-growth}C) is most readily attributable to decreasing \pCOtwo associated with evolving geological sources and sinks of carbon, modulated by the silicate weathering feedback \citep{Walker1981a, Raymo1991a, Berner1997a, Kump1997a, Berner2001a}.

\subsection{Volcanic Degassing}

\begin{figure}[h!]
\begin{center}
	\includegraphics[width=0.6\textwidth]{figures/SEAIs/weatherability-curves.pdf}
	\caption[Weatherability curves.]{Weatherability curves for the modern and paleo-SEAIs scenarios shown in Figure \ref{fig:scenario-pCO2}. The lower panel expands the lower \pCOtwo range (x-axis) of the upper panel. Details on how these curves were generated are described in \MM. Each of the 4 curves represent a different tectonic boundary condition (i.e. the reconstructed paleoshorelines of the SEAIs; Fig. \ref{fig:shoreline-growth}A) and therefore a different global weatherability. The curves show the resulting \pCOtwo for a given volcanic degassing flux such that the input flux is balanced by the silicate weathering output flux. Point B represents the pre-industrial, in which \pCOtwo is 286~ppm. The arrow from Point A$_{1}$ to B represents the ``increase in weatherability only'' scenario, in which global weatherability increases as the SEAIs emerge, but the volcanic degassing flux does not change over the past 15~m.y. In this scenario, the \pCOtwo decreases from the value dictated by the 15~Ma paleo-SEAIs weatherability curve (568~ppm). The arrow from Point A$_{2}$ to B instead represents the ``decrease in degassing only'' scenario, in which global weatherability remains the same as the pre-industrial, but the same change in \pCOtwo as the ``increase in weatherability only'' scenario is achieved by decreasing the volcanic degassing flux from a value $\sim$13\% greater than the pre-industrial. The arrow from Point A$_{3}$ to B represents the ``increase in weatherability and degassing'' scenario, in which a change in \pCOtwo from 400~ppm to 286~ppm is achieved by increasing both global weatherability from our 15~Ma tectonic boundary condition as well as the volcanic degassing flux from a value $\sim$7\% smaller than the pre-industrial flux.}
	\label{fig:weatherability-curves}
\end{center}
\end{figure}

A decrease in volcanic degassing \citep{Berner1983a} has also been proposed as a driver for Neogene cooling. However, proxy-based estimates of the evolution of volcanic degassing fluxes throughout the Neogene are inconsistent with each other, such that not even the sign of the change in volcanic degassing over the past $\sim$15~m.y. is without ambiguity \citep{Godderis2017c}. For example, it has both been estimated that the volcanic degassing flux was $\sim$25\% lower \citep{Cogne2006a} and $\sim$10\% higher \citep{Van-Der-Meer2014a} at 15~Ma relative to the present day.

Our model framework provides an opportunity to estimate the decrease in volcanic degassing flux necessary to achieve the same change in \pCOtwo predicted for the increase in global weatherability associated with the emergence of the SEAIs over the past 15~m.y. If we use the parameter combination that had the highest $r^{2}$ between computed and measured \COtwo consumption in watersheds around the world during calibration (\textit{Calibration} and \SI; Fig. \ref{fig:r2-cross-plot}), GEOCLIM estimates a pre-industrial volcanic degassing flux of 4.1$\times$10$^{12}$~mol/yr to balance the silicate weathering flux at 286~ppm \pCOtwo. If we then assume that this volcanic degassing flux did not change over the past 15~m.y., then GEOCLIM estimates that the increase in global weatherability associated with the emergence of the SEAIs led to a change in \pCOtwo of $\sim$280~ppm (``increase in weatherability only'' scenario in Fig. \ref{fig:weatherability-curves}). If we instead assume that global weatherability did not change over the past 15~m.y., then we estimate that the volcanic degassing flux needs to have been $\sim$13\% greater at 15~Ma relative to the pre-industrial to drive the same $\sim$280~ppm change in \pCOtwo (``decrease in degassing only'' scenario in Fig. \ref{fig:weatherability-curves}). This $\sim$13\% value is higher than 10\%, the highest current estimate for the volcanic degassing flux at 15~Ma relative to the present day \citep{Van-Der-Meer2014a}.

However, changes in the volcanic degassing flux would have modulated changes in \pCOtwo associated with changes in global weatherability. For example, some proxy-based approaches as well as some model-data assimilation approaches estimate that mid-Miocene \pCOtwo was lower than 568~ppm (\SI; Fig. \ref{fig:pCO2-proxies}). Take a scenario in which \pCOtwo was 400~ppm at 15~Ma. If we assume that the \pCOtwo decrease to the pre-industrial value of 286~ppm was driven by the increase in global weatherability associated with emergence of the SEAIs in conjunction with an increase in volcanic degassing which counteracts cooling by increasing the flux of \COtwo to the atmosphere (``increase in weatherability and degassing'' scenario in Fig. \ref{fig:weatherability-curves}), the volcanic degassing flux would have had to have been $\sim$7\% smaller than the pre-industrial. More robust constraints on \pCOtwo (\SI; Fig. \ref{fig:pCO2-proxies}) and/or volcanic degassing rates over the past 20~m.y. are needed to constrain which of the ``increase in weatherability only'' or ``increase in weatherability and degassing'' scenarios (Fig. \ref{fig:weatherability-curves}) is more representative of the mechanisms driving Neogene cooling.

\subsection{Himalayan Uplift}

Marine \SrSr has overall been increasing since ca. 35~Ma \citep{McArthur2012a}. The traditional explanation for this trend is that it reflects increased weathering of radiogenic (i.e. high \SrSr) silicate rocks \citep{Raymo1988a, Edmond1992a}. Associated with this explanation is the proposal that increasing weathering of radiogenic silicate rocks in the Himalayas was the primary driver of Neogene cooling \citep{Raymo1992a}. It could be argued that increasing marine \SrSr is inconsistent with the hypothesis that increasing weathering of juvenile (i.e. low \SrSr) silicate rocks in the SEAIs was an important driver of Neogene cooling. However, the globally averaged ratio of silicate weathering fluxes from radiogenic cratonic rocks versus juvenile arc lithologies can be at least partially decoupled from marine \SrSr via the regional weathering of isotopically unique lithologies. For example, in addition to highly radiogenic granites and gneisses \citep{Edmond1992a}, unusually radiogenic carbonates are abundant in Himalayan strata, and it is estimated that $\sim$75\% of Sr coming from the Himalayas can be attributed to carbonate rather than silicate weathering \citep{Jacobson2002a, Quade2003a, Oliver2003a}. As such, there are challenges in interpreting the marine \SrSr record as a direct proxy for silicate weathering fluxes. Nevertheless, steadily increasing marine \SrSr is interrupted ca. 16~Ma when the slope of the \SrSr curve decreases \citep{McArthur2012a}. This decrease in slope has been attributed to coincident exhumation of relatively low \SrSr Outer Lesser Himalaya carbonates \citep{Myrow2015a, Colleps2018a}, but could also be at least partially driven by the emergence of low \SrSr lithologies in the SEAIs during arc-continent collision. Increasing seawater Mg/Ca since ca. 15~Ma \citep{Higgins2012a} is consistent with an increasing proportion of the global silicate weathering flux being derived from mafic and ultramafic sources.

Himalayan uplift would have affected geological carbon sinks, either via increased weathering of silicate rocks \citep{Raymo1992a} or enhanced burial of organic matter in the Bengal Fan \citep{Galy2007a}. Increased weathering of the emerging SEAIs would have occurred in tandem with such changes in the Himalaya, such that the effects of these paleogeographic changes on geochemical proxy records, like marine \SrSr, become difficult to disentangle. In addition, given the large uncertainty associated with changes in regional climatology across Asia due to Himalayan orogeny, developing quantitative estimates of the evolution of global silicate weathering fluxes associated with Himalayan orogeny remains a major challenge.

\section{The Geologic Carbon Cycle}

If geological carbon sources remain approximately constant, global alkalinity delivery from silicate weathering needs to be approximately constant as well to keep the long-term carbon cycle in steady-state \citep{Kump1997a}. Enhanced silicate weathering in a region such as the SEAIs is compensated by a decrease in silicate weathering elsewhere. Global alkalinity delivery from silicate weathering does not change, but occurs more efficiently and thereby at lower \pCOtwo. Given that carbonate weathering is disconnected from the long-term carbon-cycle mass balance, changes in carbonate accumulation through time \citep{Si2019a} could be driven by changes in carbonate weathering.

The long-term carbon-cycle mass balance can be perturbed via mechanisms that are disconnected from changes in volcanic degassing and silicate weathering rates. For example, sulphide oxidation coupled to carbonate dissolution could act as a source of \COtwo on million year time-scales \citep{Torres2014a}. Similarly, the weathering of sedimentary organic matter could serve as a source of \COtwo \citep{Hilton2014a}. On the other hand, enhanced burial of organic matter enabled by higher sediment and nutrient delivery could be an important sink of \COtwo, as has been suggested in the Bengal Fan \citep{Galy2007a} and Taiwan \citep{Kao2014a}. The fluxes of \COtwo represented by these processes are not accounted for in our model framework, and could have been affected by emergence of the SEAIs and/or Himalayan orogeny. \pCOtwo changes that result from these processes would be superimposed on \pCOtwo changes associated with evolving silicate weathering fluxes. However, our coupled weathering-climate model indicates that the \pCOtwo change associated with increased global weatherability driven by emergence of the SEAIs is sufficient to explain the majority of Neogene cooling (Fig. \ref{fig:scenario-pCO2}). Without this emergence, \pCOtwo would have remained above the level necessary for the growth of Northern Hemisphere ice sheets.

\section{Conclusions}

Coupled geological constraints and modeling experiments demonstrate that the SEAIs have been a growing hot spot for carbon sequestration due to silicate weathering from the Miocene to present. Changes in volcanic degassing and paleogeography elsewhere on Earth, particularly in the Himalaya and Central America, would have also affected geological carbon sources and sinks. Yet, not only does the history of emergence of the SEAIs coincide with Neogene cooling and the onset of Northern Hemisphere glaciation, but our coupled weathering-climate model also indicates that the associated steady-state \pCOtwo change is sufficient to explain much of this cooling. These results highlight that the Earth's climate state is particularly sensitive to changes in tropical geography.

\section{Materials and Methods}

The code for the GEOCLIM model used in this study can be found at: \url{https://github.com/piermafrost/GEOCLIM-dynsoil-steady-state/releases/tag/v1.0}. The code that generated the inputs and analyzed the output of the GEOCLIM model can be found at: \url{https://github.com/Swanson-Hysell-Group/GEOCLIM_Modern}.

\subsection{GEOCLIM Silicate Weathering Component}

The silicate weathering component of the GEOCLIM model has been modified from the previously published version \citep{Godderis2017b}. The new component implements the model of \citet{Gabet2009a} for the development of a chemically weathered profile. We refer to this chemically weathered profile as regolith where the base of the regolith is unweathered bedrock. In the model of \citet{Gabet2009a}, material enters the regolith and leaves either as a solute through chemical weathering of the material during its travel from the bedrock towards the surface, or as a physically weathered particle once it reaches the top. We use the DynSoil implementation of the \citet{Gabet2009a} model, which integrates chemical weathering within the regolith \citep{West2012a}. The transient time-varying version of this regolith model is described by three equations:

\begin{equation}
    \frac{dh}{dt} = P_{r} - E_{p}
    \label{eq:1}
\end{equation}

\begin{equation}
    \frac{\partial x}{\partial t} = -P_{r} \frac{\partial x}{\partial z} - K \tau^{\sigma}x
    \label{eq:2}
\end{equation}

\begin{equation}
    \frac{\partial \tau}{\partial t} = -P_{r} \frac{\partial \tau}{\partial z} + 1
    \label{eq:3}
\end{equation}

\noindent
Equation \ref{eq:1} is a statement of material conservation, where $h$ is the total height of the regolith (m), $t$ is the model time (yr), $P_{r}$ is the regolith production rate (m/yr), and $E_{p}$ is the physical erosion rate (m/yr). Equation \ref{eq:2} describes how the residual fraction of weatherable phases ($x$, unitless) changes as a function of time ($t$, yr) and depth ($z$, m). $K \tau^{\sigma}$ is the dissolution rate constant, which depends on the local climate (captured by $K$, yr$^{-1-\sigma}$) and the time that a given rock particle has spent in the regolith ($\tau$, yr) to some power $\sigma$ (unitless) which implements a time-dependence. Equation \ref{eq:3} describes how the time that a given rock particle has spent in the regolith changes as time in the model progresses.

The net weathering rate in the regolith column ($W$, m/yr) can then be calculated with:

\begin{equation}
    W = \int_{0}^{h} K \tau^{\sigma} x\;dz
    \label{eq:4}
\end{equation}

The regolith production rate can be expressed as the product of the optimal production rate ($P_{0}$) and a soil production function ($f(h)$):

\begin{equation}
    P_{r} = P_{0}\;f(h)
    \label{eq:5}
\end{equation}

\begin{equation}
    P_{0} = k_{rp}\;q\;e^{-\frac{E_{a}}{R}\left(\frac{1}{T}-\frac{1}{T_{0}}\right)}
    \label{eq:6}
\end{equation}

\begin{equation}
    f(h) = e^{\frac{-h}{d_{0}}}
    \label{eq:7}
\end{equation}

\noindent
$P_{0}$ is the `optimal' regolith production rate (m/yr), which is defined to be the regolith production rate when there is no overlying regolith. In Equation \ref{eq:6}, where $k_{rp}$ is a proportionality constant (unitless), $q$ is the runoff (m/yr), $E_{a}$ is the activation energy (J/K/mol), $R$ is the ideal gas constant (J/mol), $T$ is the temperature (K), and $T_{0}$ is the reference temperature (K), we parameterize the `optimal' regolith production rate \citep{Carretier2014a}. $f(h)$ is the soil production function (unitless), which describes how regolith production decreases as the thickness of the regolith increases. It takes an exponential form as is commonly applied in the literature \citep{Gabet2009a}. In Equation \ref{eq:7}, $d_{0}$ is a reference regolith thickness (m) \citep{Heimsath1997a}.

Our implementation of the erosion rate is parameterized based on runoff and slope ($s$; m/m):

\begin{equation}
    E_{p} = k_{e}\;q^{m}\;s^{n}
    \label{eq:8}
\end{equation}

\noindent
$k_{e}$ is a proportionality constant ((m/yr)$^{1-m}$) and $m$ and $n$ are adjustable exponents that are kept as 0.5 and 1 \citep{Maffre2018a}. This formulation is directly inspired by the stream power law \citep{Davy2000a}. This formulation and these exponent values are supported by compilations, but variability in the proportionality constant is difficult to capture at a global scale \citep{Lague2013a}.

The $K$ in the dissolution rate constant in Equation \ref{eq:2} describes the dependence of the chemical weathering on climate:

\begin{equation}
    K = k_{d}\left(1-e^{-k_{w}q}\right)e^{-\frac{E_{a}}{R}\left(\frac{1}{T}-\frac{1}{T_{0}}\right)}
    \label{eq:9}
\end{equation}

\noindent
Equation \ref{eq:9} is an empirical simplification of mineral dissolution rates derived from kinetic theory and laboratory experiments \citep{West2012a}, where $k_{d}$ is a proportionality constant that modifies the dependence of dissolution rate on runoff and temperature (yr$^{-1-\sigma}$), and $k_{w}$ is a proportionality constant that modifies the dependence of dissolution rate on runoff (yr/m).

In this study, we are interested in obtaining the steady-state solution rather than the transient time-varying solution. The steady-state solution for DynSoil can be calculated analytically by setting the time derivatives equal to zero resulting in the following set of equations:

\begin{equation}
    h = \max\left(0,\;d_{0} \log\left(\frac{P_{0}}{E_{p}}\right)\right)
    \label{eq:10}
\end{equation}

\begin{equation}
    x(z) = \exp\left(-\frac{K}{\sigma+1}\left(\frac{z}{E_{p}}\right)^{\sigma+1}\right)
    \label{eq:11}
\end{equation}

\begin{equation}
    W = E_{p}(1-x(h)) = E_{p}(1-x_s)
    \label{eq:12}
\end{equation}

\noindent
$x(z)$ is the abundance profile of primary phases inside the regolith, varying with height upward from the base of the regolith as shown in Figure \ref{fig:regolith-schematic}. Setting $z$ equal to the regolith thickness ($h$) gives $x_s$ which is the proportion of primary phases remaining at the top of the regolith column.

\subsection{Weatherability Curves}

To create the 15~Ma paleo-SEAIs curve shown in Figure \ref{fig:weatherability-curves}, we use the reconstructed paleoshorelines of the SEAIs at 15~Ma (Fig. \ref{fig:shoreline-growth}A). We then select the parameter combination that had the highest $r^{2}$ between computed and measured \COtwo consumption in watersheds around the world during calibration (\textit{Calibration} and \SI; Fig. \ref{fig:r2-cross-plot}), and fix \pCOtwo at the 3 \pCOtwo levels at which the GFDL CM2.0 climate model experiments were computed (\SI). We then run GEOCLIM at each of these \pCOtwo levels until steady state is achieved (i.e. until the volcanic degassing flux is equal to the silicate weathering flux). We then repeat this process for the 10 and 5~Ma paleo-SEAIs paleoshorelines and the present day shorelines to generate the 3 other weatherability curves. Each estimated \pCOtwo in Figure \ref{fig:scenario-pCO2} is the result of underlying weatherability curves that change with the different chemical weathering parameters.

\subsection{Supporting Information}

A detailed description of the implementation of lithology into the silicate weathering component of GEOCLIM, the calibration of the silicate weathering component of GEOCLIM, the GFDL CM2.0 climate model, and the paleoshoreline reconstructions can be found in the \SI.

\section{Acknowledgements}

Collaborative research between N.L.S.-H. and Y.G. was initially supported by a grant from the France-Berkeley Fund. Project research was supported by NSF FRES grants \#1926001 and \#1925990. We thank Alec Brenner, Sam Lo Bianco, Mariana Lin, and Judy Pu for their data compilation contributions to the paleoshoreline reconstructions.
\include{chap-Tambien}

%%%%%%%%%%%%%%%%%%%%%%%%
%%%%% BIBLIOGRAPHY %%%%%
%%%%%%%%%%%%%%%%%%%%%%%%

\bibliography{references}

%%%%%%%%%%%%%%%%%%%%
%%%%% APPENDIX %%%%%
%%%%%%%%%%%%%%%%%%%%

\appendix
\include{SI-SEAIs}
\include{SI-Tambien}

\end{document}
